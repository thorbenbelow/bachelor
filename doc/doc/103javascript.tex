\clearpage

\chapter{\textbf{JavaScript}}\label{javascript}

JavaScript (JS) is a general purpose programming language that is best-known as the scripting language for the Web and is defined by the ECMAScript Specification.
More precisely by the ECMAScript Language Specification (ECMA-262) and the ECMAScript Internationalization API Specification (ECMA-402) \cite{mdn-js}.

\section{Core concepts}\label{js:core-concepts}
JavaScripts core concepts include being prototype-based, supporting multiple paradigms such as object-oriented and functional programming and having higher order functions \cite{mdn-js}. The following sections give a brief overview over the most important core concepts.

\subsection{Inheritance and the prototype chain}




\subsection{Higher order functions}
\subsection{Closures}
\subsection{Strict mode}
\subsection{Event Loop}
\subsection{Memory management}

\section{Compilation}


\section{Parallelism in JavaScript}
ECMAScript itself has no notion of Threads. It only has the concept of Realms and execution contexts.
Instead the Engines and envrionments that JavaScript is executed in provide methods to create new execution contexts (Threads) and interact with them.


\subsection{Browser specific}
In the Browser these are different kind of Workers.
Currently there are Web, Shared and Service workers.
The Web Worker is akin to a "normal" Thread. It lives only as long as its parent Context and can only communicate with it and contexts that are related to the parent context.

The Shared Worker allows communication with multiple "Parent" contexts. It lives as long as atleast one parent context is still alive.


The Service Worker can persist even after the parent context is already closed. It can communicate with all other contexts that have the same origin.


\subsection{Serverside}
nodejs and deno provide Worker Threads. Theses behave similar to threads known in other programming languages.

\addtocontents{toc}{\vspace{0.8cm}}
